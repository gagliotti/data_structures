\documentclass[12pt,table,xcolor={dvipsnames}]{beamer}
%\usetheme{Pittsburgh}
\usecolortheme{seagull}
%\usepackage[utf8]{inputenc}
\usepackage{fontspec}
\usepackage[brazilian]{babel}
\usepackage{amsmath}
\usepackage{listings}
\usepackage{tikz}
\usetikzlibrary{calc,shapes.multipart,chains,arrows, positioning}
\usepackage{multirow}
\usepackage{amsfonts}
\usepackage{amssymb}
%\usepackage{lstlinebgrd}
\usepackage{graphicx}
\usepackage{wasysym}
\author{Lista Encadeada}
\title{Estruturas de Dados}
%\setbeamercovered{transparent}
\setbeamertemplate{navigation symbols}{}
%\logo{\includegraphics[scale=0.015]{Brasao_UFSC.png}\includegraphics[scale=0.2]{brasao_PPGCC.jpg}}
\institute{Departamento de Computação \\ Prof. Martín Vigil \\ Adaptado de \url{}}
\date{2020.1}
\subject{}
\usebackgroundtemplate{\includegraphics[width=\paperwidth,
height=\paperheight]{../reusable_images/fundo_UFSC.png}}
\begin{document}

{
\usebackgroundtemplate{\includegraphics[width=\paperwidth,
height=\paperheight]{../reusable_images/fundo_capa.png}}
\begin{frame}
\titlepage
%\includegraphics[scale=0.3]{../reusable_images/brasao_INE.png}
\end{frame}
}

%TODO removido
\begin{frame}<0>[fragile]{Listas Usando Vetores: Desvantagens}
\begin{columns}
\column{.5\textwidth}
\begin{itemize}
\item Tamanho máximo fixo;
\item Mesmo vazias ocupam um grande espaço de memória;
\item Operações podem envolver muitos deslocamentos de dados:
\begin{itemize}
\item Inclusão em uma posição ou no início;
\item Exclusão em uma posição ou no início.
\end{itemize}
\end{itemize}
\column{.5\textwidth}
\begin{center}
\begin{tabular}{| p{.5cm} |p{.5cm}c }
\cline{1-1}
 \cellcolor{SpringGreen} 24 & &\\ \cline{1-1}
\cellcolor{GreenYellow}  89& &\\ \cline{1-1}
\cellcolor{SpringGreen}  12& &\\ \cline{1-1}
\cellcolor{GreenYellow}  4 & &\\ \cline{1-1}
\cellcolor{SpringGreen} 55 & &\\ \cline{1-2}
\cellcolor{GreenYellow} 20 & \multicolumn{1}{|p{.5cm}|} {5} & \\ \cline{1-2}
\end{tabular}
\end{center}
\end{columns}
\end{frame}

\begin{frame}[fragile]{Definição de Lista}
	\begin{itemize}
		\item É uma \textit{\textbf{sequência}} de dados
		\item Exemplos:
		\begin{itemize}
			\item $<$Pedro, João, Maria, Ivan$>$
			\item $<$5, 9, 70, 3$>$
			\item $<$\smiley, \phone, \davidsstar, \twonotes$>$
		\end{itemize}
	\end{itemize}
\end{frame}

\begin{frame}[fragile]{Definições de Lista Encadeada}
\begin{itemize}
\item É uma estrutura de dados que simula uma lista de dados
\item É composta por objetos (instâncias) do tipo \textbf{\textit{Elemento}}
\item Cada elemento 
\begin{itemize}
\item contém um dado da lista 
\item referencia o próximo elemento
\item é alocado dinamicamente somente quando necessário
\end{itemize}
\item Usa-se um objeto \textit{cabeça} da \textbf{\textit{Lista}} para:
\begin{itemize}
	\item Indicar a quantidade de elementos na lista
	\item Referenciar o primeiro elemento da Lista
\end{itemize}
%\item Para referenciar o primeiro elemento utilizamos um objeto do tipo \textbf{\textit{Lista}}.
\end{itemize}
\end{frame}


\begin{frame}<0>[fragile]{Lista Encadeada}
\setbeamercovered{invisible}
\begin{center}
\begin{tikzpicture}[list/.style={rectangle split, rectangle split parts=2,
    draw=green!60, fill=green!5, rectangle split horizontal}, >=stealth, start chain]

  \node[list,on chain] (H) {3};
  \node[list,on chain] (A) [below=of H] {melão};
  \node[list,on chain] (B) {maçã};
  \node[list,on chain] (C) {uva};
  \node[on chain,draw=red!60, fill=red!5,inner sep=6pt] (D) {null*};
  \draw[*->] let \p1 = (H.two), \p2 = (H.center) in (\x1+2,\y2+2) -- (A);
  \draw[*->] let \p1 = (A.two), \p2 = (A.center) in (\x1,\y2) -- (B);
  \draw[*->] let \p1 = (B.two), \p2 = (B.center) in (\x1,\y2) -- (C);
  \draw[*->] let \p1 = (C.two), \p2 = (C.center) in (\x1,\y2) -- (D);
\end{tikzpicture}
\end{center}
\end{frame}

\begin{frame}<0>[fragile]{Lista Encadeada}
\setbeamercovered{invisible}
\begin{center}
\begin{tikzpicture}[list/.style={rectangle split, rectangle split parts=2,
    draw=green!60, fill=green!5, rectangle split horizontal}, >=stealth, start chain]

  \node[list,on chain] (H) {3};
  \node at (2,0) (T) {\color{red}Cabeça};
  \draw[red,<-] (H) -- (T.west);
  \node[list,on chain] (A) [below=of H] {melão};
  \node[list,on chain] (B) {maçã};
  \node[list,on chain] (C) {uva};
  \node[on chain,draw=red!60, fill=red!5,inner sep=6pt] (D) {null*};
  \draw[*->] let \p1 = (H.two), \p2 = (H.center) in (\x1+2,\y2+2) -- (A);
  \draw[*->] let \p1 = (A.two), \p2 = (A.center) in (\x1,\y2) -- (B);
  \draw[*->] let \p1 = (B.two), \p2 = (B.center) in (\x1,\y2) -- (C);
  \draw[*->] let \p1 = (C.two), \p2 = (C.center) in (\x1,\y2) -- (D);
\end{tikzpicture}
\end{center}
\end{frame}

\begin{frame}<0>[fragile]{Lista Encadeada}
\setbeamercovered{invisible}
\begin{center}
\begin{tikzpicture}[list/.style={rectangle split, rectangle split parts=2,
    draw=green!60, fill=green!5, rectangle split horizontal}, >=stealth, start chain]

  \node[list,on chain] (H) {3};
  \node at (2,0) (T) {\color{red}Cabeça};
  \draw[red,<-] (H) -- (T.west);
  \node[list,on chain] (A) [below=of H] {melão};
  \node (T2)[below=of A] {\color{red}Elemento};
  \draw[red,<-] (A) -- (T2.north);
  \node[list,on chain] (B) {maçã};
  \node[list,on chain] (C) {uva};
  \node[on chain,draw=red!60, fill=red!5,inner sep=6pt] (D) {null*};
  \draw[*->] let \p1 = (H.two), \p2 = (H.center) in (\x1+2,\y2+2) -- (A);
  \draw[*->] let \p1 = (A.two), \p2 = (A.center) in (\x1,\y2) -- (B);
  \draw[*->] let \p1 = (B.two), \p2 = (B.center) in (\x1,\y2) -- (C);
  \draw[*->] let \p1 = (C.two), \p2 = (C.center) in (\x1,\y2) -- (D);
\end{tikzpicture}
\end{center}
\end{frame}

\begin{frame}[fragile]{Lista Encadeada}
\setbeamercovered{invisible}
\begin{center}
\begin{tikzpicture}[list/.style={rectangle split, rectangle split parts=2,
    draw=green!60, fill=green!5, rectangle split horizontal}, >=stealth, start chain]

  \node[list,on chain] (H) {3};
  \node at (2,0) (T) {\color{red}Lista (cabeça)};
  \draw[red,<-] (H) -- (T.west);
  \node[list,on chain] (A) [below=of H] {melão};
  \node (T2)[below=of A] {\color{red}Elemento};
  \draw[red,<-] (A) -- (T2.north);
  \node[list,on chain] (B) {maçã};
  \node[list,on chain] (C) {uva};
  \node[on chain,draw=red!60, fill=red!5,inner sep=6pt] (D) {null*};
  \node (T3)[below=of D] {\color{red}Marcador Fim};
  \draw[red,<-] (D) -- (T3.north);  
  \draw[*->] let \p1 = (H.two), \p2 = (H.center) in (\x1+2,\y2+2) -- (A);
  \draw[*->] let \p1 = (A.two), \p2 = (A.center) in (\x1,\y2) -- (B);
  \draw[*->] let \p1 = (B.two), \p2 = (B.center) in (\x1,\y2) -- (C);
  \draw[*->] let \p1 = (C.two), \p2 = (C.center) in (\x1,\y2) -- (D);
\end{tikzpicture}
\end{center}
\end{frame}


\begin{frame}[fragile]{Modelagem de Lista}
\lstset{language=C++,
          keywordstyle=\color{blue}\ttfamily,
          stringstyle=\color{red}\ttfamily,
          commentstyle=\color{OliveGreen}\ttfamily,
          breaklines=true,
          basicstyle=\ttfamily\footnotesize
          }

\begin{itemize}
\item Aspecto Estrutural:
\begin{itemize}
\item Necessitamos um ponteiro para o primeiro elemento da lista;
\item Necessitamos um inteiro para indicar quantos elementos a lista possui.
\end{itemize}
\begin{lstlisting}
estrutura Lista {
 Elemento* _primeiro;
 int _quantidade;
};
\end{lstlisting}
\end{itemize}
\end{frame}


\begin{frame}[fragile]{Modelagem do Elemento de Lista}
\lstset{language=C++,
          keywordstyle=\color{blue}\ttfamily,
          stringstyle=\color{red}\ttfamily,
          commentstyle=\color{OliveGreen}\ttfamily,
          breaklines=true,
          basicstyle=\ttfamily\footnotesize
          }

\begin{itemize}
\item Aspecto Estrutural:
\begin{itemize}
\item Necessitamos um ponteiro para o próximo elemento da lista;
\item Necessitamos um ponteiro \textbf{genérico} T* para o dado que vamos armazenar.
\item O dado necessita de um destrutor próprio, assim como a lista vai precisar de um também;
\end{itemize}
\begin{lstlisting}
estrutura Elemento {
 Elemento* _proximo;
 T* _dado;
};
\end{lstlisting}
\end{itemize}
\end{frame}

\begin{frame}<0>[fragile]{Modelagem de Lista Encadeada}
\setbeamercovered{invisible}
\begin{center}
\begin{tikzpicture}[list/.style={rectangle split, rectangle split parts=2,
    draw=green!60, fill=green!5, rectangle split horizontal}, 
    element/.style={rectangle split, rectangle, draw=green!60, fill=green!5,},    
    >=stealth, start chain]

  \node[list,on chain] (H) {3};
  \node[list,on chain] (A) [below=of H] {melão};
  \node[list,on chain] (B) {maçã};
  \node[list,on chain] (C) {uva};
  \node[on chain,draw=red!60, fill=red!5,inner sep=6pt] (D) {null*};
  \draw[*->] let \p1 = (H.two), \p2 = (H.center) in (\x1+2,\y2+2) -- (A);
  \draw[*->] let \p1 = (A.two), \p2 = (A.center) in (\x1,\y2) -- (B);
  \draw[*->] let \p1 = (B.two), \p2 = (B.center) in (\x1,\y2) -- (C);
  \draw[*->] let \p1 = (C.two), \p2 = (C.center) in (\x1,\y2) -- (D);
\end{tikzpicture}
\end{center}
\end{frame}

\begin{frame}[fragile]{Modelagem de Lista Encadeada}
\setbeamercovered{invisible}
\begin{center}
\begin{tikzpicture}[list/.style={rectangle split, rectangle split parts=2,
    draw=green!60, fill=green!5, rectangle split horizontal}, 
    element/.style={rectangle, draw=yellow!60, fill=yellow!5,},    
    >=stealth, start chain]

  \node[list,on chain] (H) {3};
  \node[list,on chain] (A) [below=of H] {};
  \draw[*->] let \p1 = (H.two), \p2 = (H.center) in (\x1+2,\y2+2) -- (A);
  \node[list,on chain] (B) {};
  \draw[*->] let \p1 = (A.two), \p2 = (A.center) in (\x1,\y2) -- (B);
  \node[list,on chain] (C) {};
  \draw[*->] let \p1 = (B.two), \p2 = (B.center) in (\x1,\y2) -- (C);
  \node[on chain,draw=red!60, fill=red!5,inner sep=6pt] (D) {null*};
  \draw[*->] let \p1 = (C.two), \p2 = (C.center) in (\x1,\y2) -- (D);
  \node[element] (AP) [below=of A] {melão};
  \draw[*->] let \p1 = (A.one), \p2 = (A.center) in (\x1+2,\y2+2) -- (AP);
  \node[element] (BP) [below=of B] {maçã};
  \draw[*->] let \p1 = (B.one), \p2 = (B.center) in (\x1+2,\y2+2) -- (BP);
  \node[element] (CP) [below=of C] {uva};
  \draw[*->] let \p1 = (C.one), \p2 = (C.center) in (\x1+2,\y2+2) -- (CP);
 
\end{tikzpicture}
\end{center}
%Para tornar todos os algoritmos da lista mais genéricos, fazemos o campo info ser um ponteiro para um elemento de informação.
\end{frame}


\begin{frame}[fragile]{Modelagem da Lista Encadeada}
\lstset{language=C++,
          keywordstyle=\color{blue}\ttfamily,
          stringstyle=\color{red}\ttfamily,
          commentstyle=\color{OliveGreen}\ttfamily,
          breaklines=true,
          basicstyle=\ttfamily\footnotesize
          }

\begin{itemize}
\item Aspecto Funcional:
\begin{itemize}
\item Temos que colocar e retirar dados da lista;
\item Temos que testar se a lista está vazia (dentre outros testes);
\item Temos que inicializar a lista e garantir a ordem de seus elementos.
\end{itemize}
\end{itemize}
\end{frame}

\begin{frame}[fragile]{Modelagem da Lista Encadeada}
\lstset{language=C++,
          keywordstyle=\color{blue}\ttfamily,
          stringstyle=\color{red}\ttfamily,
          commentstyle=\color{OliveGreen}\ttfamily,
          breaklines=true,
          basicstyle=\ttfamily\footnotesize
          }

\begin{itemize}
\item Inicializar ou limpar:
\begin{itemize}
\item Lista* iniciaLista();
%\item limpaLista(Lista* umaLista);
\item destroiLista(Lista* umaLista);
\end{itemize}
\item Testar se a lista está vazia ou cheia e outros testes:
\begin{itemize}
\item bool listaVazia(Lista* umaLista);
\item int posicao(Lista* umaLista, T* umDado);
\item bool contem(Lista* umaLista, T* umDado);
\end{itemize}
\end{itemize}
\end{frame}

\begin{frame}[fragile]{Modelagem da Lista Encadeada}
\lstset{language=C++,
          keywordstyle=\color{blue}\ttfamily,
          stringstyle=\color{red}\ttfamily,
          commentstyle=\color{OliveGreen}\ttfamily,
          breaklines=true,
          basicstyle=\ttfamily\footnotesize
          }

\begin{itemize}
\item Colocar e retirar dados da lista:
\begin{itemize}
\item adiciona(Lista* umaLista, T* umDado);
\item adicionaNoInicio(Lista* umaLista, T* umDado);
\item adicionaNaPosicao(Lista* umaLista, T* umDado, int umaPosicao);
\item adicionaNoFim(Lista* umaLista, T* umDado);
%\item adicionaEmOrdem(umaLista, umDado);
%\item T retira(umaLista);
\item T* retiraDoInicio(Lista* umaLista);
\item T* retiraDaPosicao(Lista* umaLista, int umaPosicao);
\item T* retiraDoFim(Lista* umaLista);
\item T* retiraEspecifico(Lista* umaLista, T* umDado);
\end{itemize}
\end{itemize}
\end{frame}

\begin{frame}[fragile]{Função \textit{iniciaLista}}
\lstset{language=C++,
          keywordstyle=\color{blue}\ttfamily,
          stringstyle=\color{red}\ttfamily,
          commentstyle=\color{OliveGreen}\ttfamily,
          breaklines=true,
          basicstyle=\ttfamily\footnotesize
          }

\begin{itemize}
\item Inicializamos o ponteiro para nulo;
\item Inicializamos o tamanho para ``0'';
\item Complexidade de tempo: $\Theta(1)$
\end{itemize}
\begin{lstlisting}
iniciaLista()
inicio
  Lista* umaLista <- ALOQUE(Lista);
  umaLista._primeiro <- null;
  umaLista._quantidade <- 0;
  RETORNE umaLista;
fim;
\end{lstlisting}
\end{frame}


\begin{frame}[fragile]{Função \textit{listaVazia}}
\lstset{language=C++,
          keywordstyle=\color{blue}\ttfamily,
          stringstyle=\color{red}\ttfamily,
          commentstyle=\color{OliveGreen}\ttfamily,
          breaklines=true,
          basicstyle=\ttfamily\footnotesize
          }

\begin{lstlisting}
bool listaVazia(Lista* umaLista)
inicio
SE (umaLista._quantidade = 0) ENTAO
 RETORNE(Verdadeiro)
SENAO
 RETORNE(Falso);
fim;
\end{lstlisting}
%\begin{itemize}
%\item Um algoritmo ListaCheia não existe na Lista Encadeada;
%\item Verificar se houve espaço na memória para um novo elemento será responsabilidade de cada operação de adição.
%\end{itemize}
\begin{itemize}
	\item Complexidade de tempo: $\Theta(1)$
\end{itemize}
\end{frame}

\begin{frame}[fragile]{Função \textit{adicionaNoInicio}}
\setbeamercovered{invisible}
\begin{enumerate}
\item Teste se é possível alocar um novo elemento;
\item Faça o próximo do novo elemento ser o primeiro da lista;
\item Faça a cabeça de lista apontar para o novo elemento.
\end{enumerate}
\begin{center}
\begin{tikzpicture}[list/.style={rectangle split, rectangle split parts=2,
    draw=green!60, fill=green!5, rectangle split horizontal}, 
    element/.style={rectangle, draw=yellow!60, fill=yellow!5,},    
    >=stealth, start chain]

  \node<5->[list] (H) {3};
  \node<1-4>[list,on chain] (H) {2};

  \node<2->[list,draw=blue!60, fill=blue!5,on chain] (N) [below left=of H] {Novo}; 
  \draw<2->[*->] let \p1 = (N.two), \p2 = (N.center) in (\x1,\y2) -- (A);  
  
  \node<2->[list,on chain] (A) [below=of H] {};
  \draw<2->[*->] let \p1 = (A.one), \p2 = (A.center) in (\x1+2,\y2+2) -- (AP);
  \node<2->[element] (AP) [below=of A] {melão};
  \draw<3->[*->] let \p1 = (A.two), \p2 = (A.center) in (\x1,\y2) -- (B);
  
  \node<1>[list,on chain] (B) [below right=of H] {};
  \node<2->[list,on chain] (B) {};
  
  \draw<1-3>[*->] let \p1 = (H.two), \p2 = (H.center) in (\x1+2,\y2+2) -- (B);
  \draw<4->[*->] let \p1 = (H.two), \p2 = (H.center) in (\x1+2,\y2+2) -- (A);
  
  
  \node[list,on chain] (C) {};
  \draw[*->] let \p1 = (B.two), \p2 = (B.center) in (\x1,\y2) -- (C);
  \node[on chain,draw=red!60, fill=red!5,inner sep=6pt] (D) {null*};
  \draw[*->] let \p1 = (C.two), \p2 = (C.center) in (\x1,\y2) -- (D);
  
  \node[element] (BP) [below=of B] {maçã}; 
  \draw[*->] let \p1 = (B.one), \p2 = (B.center) in (\x1+2,\y2+2) -- (BP);
  \node[element] (CP) [below=of C] {uva};
  \draw[*->] let \p1 = (C.one), \p2 = (C.center) in (\x1+2,\y2+2) -- (CP);
 
\end{tikzpicture}
\end{center}
\end{frame}

\begin{frame}[fragile]{Função \textit{adicionaNoInicio}}
\lstset{language=C++,
          keywordstyle=\color{blue}\ttfamily,
          stringstyle=\color{red}\ttfamily,
          commentstyle=\color{OliveGreen}\ttfamily,
          breaklines=true,
          escapeinside={(*@}{@*)},
          basicstyle=\ttfamily\footnotesize
          }
\begin{lstlisting}
adicionaNoInicio(Lista* umaLista, T* umDado)
 inicio
  Elemento* novo <- ALOQUE(Elemento); 
  SE (novo = NULO) ENTAO
   THROW(ERROSEMMEMORIA);
  SENAO
   novo._proximo <- umaLista._primeiro;
   novo._dado <- umDado;
   umaLista._primeiro <- novo;
   umaLista._quantidade <- umaLista._quantidade + 1;
  FIM SE
  fim;
(*@  \textcolor{red}{}  @*)
\end{lstlisting}
\begin{itemize}
	\item Complexidade de tempo: ?
\end{itemize}
\end{frame}

\begin{frame}<0>[fragile]{Função \textit{adicionaNoInicio}}
\lstset{language=C++,
          keywordstyle=\color{blue}\ttfamily,
          stringstyle=\color{red}\ttfamily,
          commentstyle=\color{OliveGreen}\ttfamily,
          breaklines=true,
          escapeinside={(*@}{@*)},
          basicstyle=\ttfamily\footnotesize
          }
\begin{lstlisting}
adicionaNoInicio(T dado)
 Elemento *novo; //Variável auxiliar.
 início
  (*@  \textcolor{red}{novo <- aloque(Elemento);}  @*)
  (*@  \textcolor{red}{SE (novo = NULO) ENTAO}  @*)
   (*@  \textcolor{red}{THROW(ERROLISTACHEIA);}  @*)
  SENAO
   novo._proximo <- _primeiro;
   novo._info <- dado;
   _primeiro <- novo;
   _quantidade <- _quantidade + 1;
  FIM SE
  fim;
(*@  \textcolor{red}{}  @*)
\end{lstlisting}
\end{frame}


\begin{frame}[fragile]{Função \textit{retiraDoInicio}}
\setbeamercovered{invisible}
\begin{itemize}
\item Verifique que a lista não está vazia;
\item Decremente o tamanho;
\item Faça a lista apontar para o segundo elemento
\item Desaloque a memória do primeiro elemento;
\item Devolva o dado retirado.
\end{itemize}
\begin{center}
\begin{tikzpicture}[list/.style={rectangle split, rectangle split parts=2,
    draw=green!60, fill=green!5, rectangle split horizontal}, 
    element/.style={rectangle, draw=yellow!60, fill=yellow!5,},    
    >=stealth, start chain]

  \node<1-2>[list] (H) {3};
  \node<3->[list,on chain] (H) {2};

  \node<2-4>[list,draw=blue!60, fill=blue!5,on chain] (S) [below left=of H] {Saiu}; 
  \draw<2-4>[*->] let \p1 = (S.two), \p2 = (S.center) in (\x1,\y2) -- (A);  
  
  \node<2->[list,draw=blue!60, fill=blue!5,on chain] (V) [left=of AP] {Volta}; 
  \draw<2->[*->] let \p1 = (V.two), \p2 = (V.center) in (\x1,\y2) -- (AP);  
  
  \node<1-3>[list,on chain] (A) [below=of H] {};
  \draw<1-3>[*->] let \p1 = (A.one), \p2 = (A.center) in (\x1+2,\y2+2) -- (AP);
  \node<1-4>[element] (AP) [below=of A] {melão};
  \draw<1-3>[*->] let \p1 = (A.two), \p2 = (A.center) in (\x1,\y2) -- (B);
  
  \node<3->[list,on chain] (B) [below right=of H] {};
  \node<1-2>[list,on chain] (B) {};
  
  \draw<3->[*->] let \p1 = (H.two), \p2 = (H.center) in (\x1+2,\y2+2) -- (B);
  \draw<1-2>[*->] let \p1 = (H.two), \p2 = (H.center) in (\x1+2,\y2+2) -- (A);
  
  
  \node[list,on chain] (C) {};
  \draw[*->] let \p1 = (B.two), \p2 = (B.center) in (\x1,\y2) -- (C);
  \node[on chain,draw=red!60, fill=red!5,inner sep=6pt] (D) {null*};
  \draw[*->] let \p1 = (C.two), \p2 = (C.center) in (\x1,\y2) -- (D);
  
  \node[element] (BP) [below=of B] {maçã}; 
  \draw[*->] let \p1 = (B.one), \p2 = (B.center) in (\x1+2,\y2+2) -- (BP);
  \node[element] (CP) [below=of C] {uva};
  \draw[*->] let \p1 = (C.one), \p2 = (C.center) in (\x1+2,\y2+2) -- (CP);
 
\end{tikzpicture}
\end{center}
\end{frame}

\begin{frame}[fragile]{Função \textit{retiraDoInicio}}
\lstset{language=C++,
          keywordstyle=\color{blue}\ttfamily,
          stringstyle=\color{red}\ttfamily,
          commentstyle=\color{OliveGreen}\ttfamily,
          breaklines=true,
          escapeinside={(*@}{@*)},
          basicstyle=\ttfamily\footnotesize
          }
\begin{lstlisting}
T* retiraDoInicio(Lista* umaLista)
 Elemento* saiu; //Variável auxiliar elemento.
 T* volta; //Variável auxiliar tipo T.
  início
   SE (listaVazia(umaLista)) ENTAO
    THROW(ERROLISTAVAZIA);
   SENAO
    saiu <- umaLista._primeiro;
    volta <- saiu._dado;
    umaLista._primeiro <- saiu._proximo;
    umaLista._quantidade <- umaLista._quantidade - 1;
    DESALOQUE(saiu);
    RETORNE(volta);
   FIM SE
fim;
(*@  \textcolor{red}{}  @*)
\end{lstlisting}
\begin{itemize}
	\item Complexidade de tempo: ?
\end{itemize}
\end{frame}

\begin{frame}<0>[fragile]{Função \textit{retiraDoInicio}}
\lstset{language=C++,
          keywordstyle=\color{blue}\ttfamily,
          stringstyle=\color{red}\ttfamily,
          commentstyle=\color{OliveGreen}\ttfamily,
          breaklines=true,
          escapeinside={(*@}{@*)},
          basicstyle=\ttfamily\footnotesize
          }
\begin{lstlisting}
T* retiraDoInicio()
 Elemento *saiu; //Variável auxiliar elemento.
 T* volta; //Variável auxiliar tipo T.
	início
		(*@  \textcolor{red}{SE (listaVazia()) ENTAO}  @*)
		  (*@  \textcolor{red}{THROW(ERROLISTAVAZIA);}  @*)
		(*@  \textcolor{red}{SENAO}  @*)
		  saiu <- _primeiro;
		  volta <- saiu.info;
		  _primeiro <- saiu.próximo;
		  _quantidade <- _quantidade - 1;
		  DESALOQUE(saiu);
		  RETORNE(volta);
		FIM SE
	fim;
(*@  \textcolor{red}{}  @*)
\end{lstlisting}
\end{frame}

\begin{frame}<0>[fragile]{Função \textit{T retiraDoInicio()}}
\lstset{language=C++,
          keywordstyle=\color{blue}\ttfamily,
          stringstyle=\color{red}\ttfamily,
          commentstyle=\color{OliveGreen}\ttfamily,
          breaklines=true,
          escapeinside={(*@}{@*)},
          basicstyle=\ttfamily\footnotesize
          }
\begin{lstlisting}
T retiraDoInicio()
 Elemento *saiu; //Variável auxiliar elemento.
 T *volta; //Variável auxiliar tipo T.
	início
		SE (listaVazia()) ENTAO
		  THROW(ERROLISTAVAZIA);
		SENAO
		  (*@  \textcolor{red}{saiu <- \_primeiro;}  @*)
		  (*@  \textcolor{red}{volta <- saiu.info;}  @*)
		  (*@  \textcolor{red}{\_primeiro <- saiu.próximo;}  @*)
		  (*@  \textcolor{red}{\_quantidade <- \_quantidade - 1;}  @*)
		  DESALOQUE(saiu);
		  RETORNE(volta);
		FIM SE
	fim;
(*@  \textcolor{red}{}  @*)
\end{lstlisting}
\end{frame}

\begin{frame}<0>[fragile]{Função \textit{T retiraDoInicio()}}
\lstset{language=C++,
          keywordstyle=\color{blue}\ttfamily,
          stringstyle=\color{red}\ttfamily,
          commentstyle=\color{OliveGreen}\ttfamily,
          breaklines=true,
          escapeinside={(*@}{@*)},
          basicstyle=\ttfamily\footnotesize
          }
\begin{lstlisting}
T retiraDoInicio()
 Elemento *saiu; //Variável auxiliar elemento.
 T *volta; //Variável auxiliar tipo T.
	início
		SE (listaVazia()) ENTAO
		  THROW(ERROLISTAVAZIA);
		SENAO
		  saiu <- _primeiro;
		  volta <- saiu.info;
		  _primeiro <- saiu.próximo;
		  _quantidade <- _quantidade - 1;
		  (*@  \textcolor{red}{DESALOQUE(saiu);}  @*)
		  (*@  \textcolor{red}{RETORNE(volta);}  @*)
		FIM SE
	fim;
(*@  \textcolor{red}{}  @*)
\end{lstlisting}
\end{frame}

\begin{frame}<0>[fragile]{Função \textit{eliminaDoInicio()}}
\lstset{language=C++,
          keywordstyle=\color{blue}\ttfamily,
          stringstyle=\color{red}\ttfamily,
          commentstyle=\color{OliveGreen}\ttfamily,
          breaklines=true,
          escapeinside={(*@}{@*)},
          basicstyle=\ttfamily\footnotesize
          }
\begin{lstlisting}
(*@  \textcolor{red}{eliminaDoInicio()}  @*)
 Elemento *saiu; //Variável auxiliar elemento.
	início
		SE (listaVazia()) ENTAO
		  THROW(ERROLISTAVAZIA);
		SENAO
		  saiu <- _primeiro;
		  volta <- saiu.info;
		  _primeiro <- saiu.próximo;
		  _quantidade <- _quantidade - 1;
		  DESALOQUE(saiu);
		  DESALOQUE(saiu.info);
		FIM SE
	fim;
(*@  \textcolor{red}{}  @*)
\end{lstlisting}
\end{frame}

\begin{frame}<0>[fragile]{Algoritmo eliminaDoInicio()}
\lstset{language=C++,
          keywordstyle=\color{blue}\ttfamily,
          stringstyle=\color{red}\ttfamily,
          commentstyle=\color{OliveGreen}\ttfamily,
          breaklines=true,
          basicstyle=\ttfamily\footnotesize
          }
\begin{itemize}
\item Observe que a linha DESALOQUE(saiu.info) possui um perigo:
\begin{itemize}
\item Se o T for por sua vez um conjunto estruturado de dados com referências internas através de ponteiros (outra lista, por exemplo), a chamada à função DESALOQUE(saiu.info) só liberará o primeiro nível da estrutura (aquele apontado diretamente);
\item Tudo o que for referenciado através de ponteiros em info permanecerá em algum lugar da memória, provavelmente inatingível (garbage);
\item Para evitar isto pode-se criar uma função destrói(info) para o T que será chamada no lugar de DESALOQUE.
\end{itemize}
\end{itemize}
\end{frame}

\begin{frame}<0>[fragile]{Importância do Destrutor}
\lstset{language=C++,
          keywordstyle=\color{blue}\ttfamily,
          stringstyle=\color{red}\ttfamily,
          commentstyle=\color{OliveGreen}\ttfamily,
          breaklines=true,
          basicstyle=\ttfamily\footnotesize
          }
\begin{itemize}
\item O destrutor diz como o objeto será destruído quando sair de escopo;
\item No mínimo deve liberar a memória que foi alocada por chamadas “new”no construtor;
\item Se nenhum destrutor for declarado será gerado um default, que aplicará o destrutor correspondente a cada dado da classe;
\item A recursão tem que ser garantida pelo objeto.
\end{itemize}
\end{frame}

\begin{frame}[fragile]{Função \textit{adicionaNaPosicao}}
\setbeamercovered{invisible}
\begin{itemize}
\item Testamos se a posição existe e se é possível alocar;
\item Caminhamos até a posição;
\item Adicionamos o novo dado na posição;
\item Incrementamos o tamanho.
\end{itemize}
\begin{center}
\begin{tikzpicture}[list/.style={rectangle split, rectangle split parts=2,
    draw=green!60, fill=green!5, rectangle split horizontal}, 
    element/.style={rectangle, draw=yellow!60, fill=yellow!5,},    
    >=stealth, start chain]

  %cabeca
  \node<7->[list] (H) {3};
  \node<1-6>[list,on chain] (H) {2};
  \draw[*->] let \p1 = (H.two), \p2 = (H.center) in (\x1+2,\y2+2) -- (A);

  %novo
  \node<3->[list,draw=blue!60, fill=blue!5,on chain] (N) [right=of H] {Novo}; 
  \draw<4->[*->] let \p1 = (N.two), \p2 = (N.center) in (\x1+2,\y2+2) -- (B);  
  
  %anterior
  \node<2->[list,draw=blue!60, fill=blue!5,on chain] (AN) [left=of A] {Anterior}; 
  \draw<2->[*->] let \p1 = (AN.two), \p2 = (AN.center) in (\x1,\y2) -- (A);   
  
  %primeiro elemento
  \node[list,on chain] (A) [below=of H] {};
  \node<4->[list,on chain] (B) {};
  \node[list,on chain] (C) {};
  %dados 
  \draw[*->] let \p1 = (A.one), \p2 = (A.center) in (\x1+2,\y2+2) -- (AP);
  \node[element] (AP) [below=of A] {melão};
  %ponteiros
  \draw<6->[*->] let \p1 = (A.two), \p2 = (A.center) in (\x1,\y2) -- (B);
  \draw<1-2>[*->] let \p1 = (A.two), \p2 = (A.center) in (\x1,\y2) -- (C);
  \draw<3-5>[*->] let \p1 = (A.two), \p2 = (A.center) in (\x1+2,\y2-2) .. controls +(north:10mm) and +(up:10mm)  .. (C.north);
  
  
   %segundo elemento
  %dados
  \draw<4->[*->] let \p1 = (B.one), \p2 = (B.center) in (\x1+2,\y2+2) -- (BP);
  \node<4->[element] (BP) [below=of B] {maçã}; 
  %ponteiro
  \draw<5->[*->] let \p1 = (B.two), \p2 = (B.center) in (\x1,\y2) -- (C);
  
  %terceiro elemento  
  
  %dados
  \node[element] (CP) [below=of C] {uva};
  \draw[*->] let \p1 = (C.one), \p2 = (C.center) in (\x1+2,\y2+2) -- (CP);
   %null pointer
  \node[on chain,draw=red!60, fill=red!5,inner sep=6pt] (D) {null*};
  %ponteiro
  \draw[*->] let \p1 = (C.two), \p2 = (C.center) in (\x1,\y2) -- (D);
  
\end{tikzpicture}
\end{center}
\end{frame}

\begin{frame}[fragile]{Função \textit{adicionaNaPosicao}}
\lstset{language=C++,
          keywordstyle=\color{blue}\ttfamily,
          stringstyle=\color{red}\ttfamily,
          commentstyle=\color{OliveGreen}\ttfamily,
          breaklines=true,
          escapeinside={(*@}{@*)},
          basicstyle=\ttfamily\scriptsize
          }
\begin{lstlisting}
adicionaNaPosicao(Lista* umaLista, T* umDado, int umaPosicao)
 Elemento *novo, *anterior; // auxiliares.
 inicio
  SE (posicao > umaLista._quantidade + 1) ENTAO THROW(ERROPOSICAO);
  SENAO
   SE (posicao = 1) ENTAO RETORNE(adicionaNoInício(umaLista, umDado));
   SENAO
    novo <- ALOQUE(Elemento);
    SE (novo = NULO) ENTÃO THROW(ERROLISTACHEIA);
    SENAO
     anterior <- umaLista_primeiro;
     REPITA (posicao - 2) VEZES
      anterior <- anterior._proximo;
     novo._proximo <- anterior._proximo;
     novo._dado <- umDado;
     anterior._proximo <- novo;
     umaLista._quantidade <- umaLista._quantidade + 1;
    FIM SE
   FIM SE
  FIM SE
fim;
(*@  \textcolor{red}{}  @*)
\end{lstlisting}
\end{frame}


\begin{frame}[fragile]{Função \textit{retiraDaPosicao}}
\setbeamercovered{invisible}
\begin{itemize}
\item Testamos se a posição existe;
\item Caminhamos até a posição;
\item Retiramos o dado da posição;
\item Decrementamos o tamanho.
\end{itemize}
\begin{center}
\begin{tikzpicture}[list/.style={rectangle split, rectangle split parts=2,
    draw=green!60, fill=green!5, rectangle split horizontal}, 
    element/.style={rectangle, draw=yellow!60, fill=yellow!5,},    
    >=stealth, start chain]

  %cabeca
  \node<1-5>[list] (H) {3};
  \node<6->[list,on chain] (H) {2};
  \draw<1->[*->] let \p1 = (H.two), \p2 = (H.center) in (\x1+2,\y2+2) -- (A);

  %volta
  \node<3->[list,draw=blue!60, fill=blue!5,on chain] (V) [left=of AP] {Volta}; 
  \draw<3->[*->] let \p1 = (V.two), \p2 = (V.center) in (\x1+2,\y2+2) .. controls +(south:10mm) and +(down:10mm)  .. (BP);  
  
  %anterior
  \node<2->[list,draw=blue!60, fill=blue!5,on chain] (AN) [left=of A] {Anterior}; 
  \draw<2->[*->] let \p1 = (AN.two), \p2 = (AN.center) in (\x1,\y2) -- (A);   
  
  %eliminar
  \node<4->[list,draw=blue!60, fill=blue!5,on chain] (E) [right=of H] {Eliminar}; 
  \draw<4-6>[*->] let \p1 = (E.two), \p2 = (E.center) in (\x1+2,\y2) -- (B);    
  
  %primeiro elemento
  \node[list,on chain] (A) [below=of H] {};
  \node<1-6>[list,on chain] (B) {};
  \node[list,on chain] (C) {};
  %dados 
  \draw[*->] let \p1 = (A.one), \p2 = (A.center) in (\x1+2,\y2+2) -- (AP);
  \node[element] (AP) [below=of A] {melão};
  %ponteiros
  \draw<1-3>[*->] let \p1 = (A.two), \p2 = (A.center) in (\x1,\y2) -- (B);
  \draw<5->[*->] let \p1 = (A.two), \p2 = (A.center) in (\x1+2,\y2-2) .. controls +(north:10mm) and +(up:10mm)  .. (C.north);

  
   %segundo elemento
  %dados
  \draw<1-6>[*->] let \p1 = (B.one), \p2 = (B.center) in (\x1+2,\y2+2) -- (BP);
  \node[element] (BP) [below=of B] {maçã}; 
  %ponteiro
  \draw[*->] let \p1 = (B.two), \p2 = (B.center) in (\x1,\y2) -- (C);
  
  %terceiro elemento  
  
  %dados
  \node[element] (CP) [below=of C] {uva};
  \draw[*->] let \p1 = (C.one), \p2 = (C.center) in (\x1+2,\y2+2) -- (CP);
   %null pointer
  \node[on chain,draw=red!60, fill=red!5,inner sep=6pt] (D) {null*};
  %ponteiro
  \draw[*->] let \p1 = (C.two), \p2 = (C.center) in (\x1,\y2) -- (D);
  
\end{tikzpicture}
\end{center}
\end{frame}

\begin{frame}[fragile]{Função \textit{retiraDaPosicao(posicao)}}
\lstset{language=C++,
          keywordstyle=\color{blue}\ttfamily,
          stringstyle=\color{red}\ttfamily,
          commentstyle=\color{OliveGreen}\ttfamily,
          breaklines=true,
          escapeinside={(*@}{@*)},
          basicstyle=\ttfamily\scriptsize
          }
\begin{lstlisting}
T* retiraDaPosicao(Lista *umaLista, int umaPosicao)
 Elemento *anterior, *eliminar; //Variáveis elemento.
 T* volta; //Variável tipo T.
 inicio
  SE (posicao > umaLista._quantidade) ENTAO THROW(ERROPOSICAO);
  SENAO
   SE (posicao = 1) ENTAO RETORNE(retiraDoInicio(umaLista));
   SENAO
    anterior <- umaLista._primeiro;
    REPITA (umaPosicao - 2) VEZES
     anterior <- anterior._proximo;
    eliminar <- anterior._proximo;
    volta <- eliminar._dado;
    anterior._proximo <- eliminar._proximo;
    umaLista._quantidade <- umaLista._quantidade - 1;
    DESALOQUE(eliminar);
    RETORNE(volta);
   FIM SE
  FIM SE
 fim;
(*@  \textcolor{red}{}  @*)
\end{lstlisting}
\end{frame}


\begin{frame}<0>[fragile]{Função \textit{adicionaEmOrdem(T dado)}}
\setbeamercovered{invisible}
\begin{itemize}
\item Necessitamos de uma função para comparar os dados (operator::>);
\item Procuramos pela posição onde inserir comparando dados;
\item Chamamos adicionaNaPosicao(T dado, int posicao).
\end{itemize}
\begin{block}{Dica!}
Podemos implementar um versão polimórfica de adicionaNaPosicao(T dado, int posicao) que é adicionaNaPosicao(T dado, Elemento* posicao)
\end{block}
\end{frame}

\begin{frame}<0>[fragile]{Função \textit{adicionaEmOrdem(T dado)}}
\lstset{language=C++,
          keywordstyle=\color{blue}\ttfamily,
          stringstyle=\color{red}\ttfamily,
          commentstyle=\color{OliveGreen}\ttfamily,
          breaklines=true,
          escapeinside={(*@}{@*)},
          basicstyle=\ttfamily\scriptsize
          }
\begin{lstlisting}
adicionaEmOrdem(T dado)
 Elemento *atual; //Variável para caminhar.
 int posicao; // Posicao de Insercao.
 inicio
  SE (listaVazia()) ENTAO RETORNE(adicionaNoInicio(dado));
  SENAO
   atual <- _primeiro;
   posicao <- 1;
   ENQUANTO (atual._proximo ~= NULO E
             dado > atual._dado)) FACA 
    //Encontrar posição para inserir.
    atual <- atual._proximo;
    posicao <- posicao + 1;
   FIM ENQUANTO
   SE (dado > atual._dado) ENTAO //Parou porque acabou a lista.
    RETORNE(adicionaNaPosicao(dado, posicao + 1));
   SENAO
    RETORNE(adicionaNaPosicao(dado, posicao));
   FIM SE
  FIM SE
 fim;
(*@  \textcolor{red}{}  @*)
\end{lstlisting}
\end{frame}


\begin{frame}[fragile]{Função \textit{destroiLista()}}
	\begin{itemize}
		\item A lista precisa estar vazia
		\item Dúvida: por que não escrever uma única função em C que desaloque todos os elementos, dados e a lista?
	\end{itemize}
\end{frame}


\begin{frame}<0>[fragile]{Função \textit{destroiLista()}}
	\lstset{language=C++,
		keywordstyle=\color{blue}\ttfamily,
		stringstyle=\color{red}\ttfamily,
		commentstyle=\color{OliveGreen}\ttfamily,
		breaklines=true,
		escapeinside={(*@}{@*)},
		basicstyle=\ttfamily\scriptsize
	}
	\begin{lstlisting}
	destroiLista(umaLista)
	Elemento *atual, *anterior; //Variável auxiliar para caminhar.
	inicio
	SE (listaVazia(umaLista)) ENTAO THROW(ERROLISTAVAZIA);
	SENAO
	atual <- umaLista._primeiro;    
	ENQUANTO (atual != NULO) FACA 
	//Eliminar até o fim.
	anterior <- atual;
	//Vou para o próximo mesmo que seja nulo.
	atual <- atual._proximo;
	//Liberar primeiro a Info.
	DESALOQUE(anterior._dado);
	//Liberar o elemento que acabei de visitar.
	DESALOQUE(anterior);
	FIM ENQUANTO
	FIM SE
	fim;
	(*@  \textcolor{red}{}  @*)
	\end{lstlisting}
\end{frame}


\begin{frame}<0>[fragile]{Função \textit{\~{}Lista}}
	\lstset{language=C++,
		keywordstyle=\color{blue}\ttfamily,
		stringstyle=\color{red}\ttfamily,
		commentstyle=\color{OliveGreen}\ttfamily,
		breaklines=true,
		basicstyle=\ttfamily\footnotesize
	}
	
	\begin{itemize}
		\item Chamamos DestroiLista();
	\end{itemize}
	\begin{lstlisting}
	~Lista()
	inicio
	DestroiLista();
	fim;
	\end{lstlisting}
\end{frame}


\begin{frame}<0>[fragile]{Função \textit{destroiLista()}}
\lstset{language=C++,
          keywordstyle=\color{blue}\ttfamily,
          stringstyle=\color{red}\ttfamily,
          commentstyle=\color{OliveGreen}\ttfamily,
          breaklines=true,
          escapeinside={(*@}{@*)},
          basicstyle=\ttfamily\scriptsize
          }
\begin{lstlisting}
destroiLista(umaLista)
 Elemento *atual, *anterior; //Variável auxiliar para caminhar.
 inicio
  SE (listaVazia(umaLista)) ENTAO THROW(ERROLISTAVAZIA);
  SENAO
   atual <- umaLista._primeiro;    
   ENQUANTO (atual != NULO) FACA 
    //Eliminar até o fim.
    anterior <- atual;
    //Vou para o próximo mesmo que seja nulo.
    atual <- atual._proximo;
    //Liberar primeiro a Info.
    DESALOQUE(anterior._dado);
    //Liberar o elemento que acabei de visitar.
    DESALOQUE(anterior);
   FIM ENQUANTO
  FIM SE
 fim;
(*@  \textcolor{red}{}  @*)
\end{lstlisting}
\end{frame}


\begin{frame}[fragile]{Algoritmos Restantes - Por conta do aluno}
	\setbeamercovered{invisible}
	\begin{itemize}
		%\item limpaLista(Lista* lista, T* dado);
		\item destroiLista(Lista* lista);
		\item posicao(Lista* lista, T* dado);
		\item contem(Lista* lista, T* dado);

		\item adicionaNoFim(Lista* lista, T* dado)

		\item retiraDoFim(Lista* lista, T* dado)
		\item retiraEspecifico(Lista * lista, T* dado);
	\end{itemize}
\end{frame}





















\begin{frame}<0>[fragile]{Trabalho Lista Encadeada}
\lstset{language=C++,
          keywordstyle=\color{blue}\ttfamily,
          stringstyle=\color{red}\ttfamily,
          commentstyle=\color{OliveGreen}\ttfamily,
          breaklines=true,
          basicstyle=\ttfamily\footnotesize
          }
\begin{itemize}
\item Implemente uma classe Lista todas as operações vistas;
\item Implemente a lista usando Templates;
\item Use as melhores práticas de orientação a objetos;
\item Documente todas as classes, funções e atributos;
\item Aplique os testes unitários disponíveis no moodle da disciplina para validar sua estrutura de dados;
\item Entregue até a data definida no moodle.
\end{itemize}
\end{frame}

{
\usebackgroundtemplate{\includegraphics[width=\paperwidth,
height=\paperheight]{../reusable_images/fundo_capa.png}}
\begin{frame}

{\LARGE Perguntas????}

\end{frame}
}


{
\usebackgroundtemplate{\includegraphics[width=\paperwidth,
height=\paperheight]{../reusable_images/fundo_capa.png}}
\begin{frame}
\includegraphics[scale=0.8]{../reusable_images/cc_logo_arge.png}\hspace{0.5cm}
\includegraphics[scale=0.95]{../reusable_images/by.png}

\vspace{1cm}
Este trabalho está licenciado sob uma Licença Creative Commons Atribuição 4.0 Internacional. Para ver uma cópia desta licença, visite http://creativecommons.org/licenses/by/4.0/.

\end{frame}
}
\end{document}
